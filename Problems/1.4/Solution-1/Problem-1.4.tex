\documentclass{article}


\usepackage{fancyhdr}
\usepackage{amssymb}
\pagestyle{fancy}
\lhead{An introduction to the analysis of algorithms\break Michael Soltys}
\rhead{Problem 1.4 Solution\break Ryan McIntyre}
\renewcommand{\headrulewidth}{0.4pt}
\renewcommand{\headheight}{24pt}
\pagenumbering{gobble}

\newtheorem{inner}{Problem}
\newenvironment{prb}[1]
	{\renewcommand\theinner{#1}\inner}
	{\endinner}
\newcommand{\Z}{\mathbb{Z}}
\newcommand{\N}{\mathbb{N}}

\begin{document}
\begin{prb}{1.4}
Suppose that the precondition (of Algorithm 1.1) is changed to say:
$$
x\ge0 \wedge y>0\wedge x,y\in\Z
$$
where $\Z=\{\dots,-2,-1,0,1,2,\dots\}$. Is the Algorithm still correct 
in this case?
\end{prb}

\noindent{\bf Solution:}\break
The original precondition (with which the algorithm is correct) is:
$$
x\ge0\wedge y>0\wedge x,y\in\N
$$
where $\N=\{0,1,2,\dots\}$. So our work has already been done for us; 
any member of $\Z$ which is $\ge0$ is also in $\N$ (and any member of $\N$
is in $\Z$), so these preconditions are equivalent. Given that the 
algorithm was correct under the original precondition, it is also
correct under the new one.
\end{document}
