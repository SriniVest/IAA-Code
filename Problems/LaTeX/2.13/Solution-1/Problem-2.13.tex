\documentclass{article}

\usepackage{fancyhdr}
\usepackage{amsmath,amssymb,amsthm}
\usepackage[utf8]{inputenc}
\usepackage[T1]{fontenc}
\usepackage{enumitem}
\pagestyle{fancy}
\lhead{An introduction to the analysis of algorithms\break Michael Soltys}
\rhead{Problem 2.13 Solution\break Ryan McIntyre}
\renewcommand{\headrulewidth}{0.4pt}
\renewcommand{\headheight}{24pt}
\newcommand{\opt}{\text{\sc Opt}}
\pagenumbering{gobble}

\newtheorem{inner}{Problem}
\newenvironment{prb}[1]
	{\renewcommand\theinner{#1}\inner}
	{\endinner}

\begin{document}

\begin{prb}{2.13}
Suppose that edge $e_1$ has a smaller cost than any of the other
edges in graph $G$; that is, $c(e_1)<c(e_i)$, for all $i>1$. Show
that there is at least one MCST for $G$ that includes $e_1$. 
\end{prb}

\begin{proof}
First, note that if we give $G$ to Kruskal's, with the edges in
the order of their indices as opposed to sorted by weight, the 
resulting tree will include $e_1$ - a cycle cannot be formed with
the first (or second) edge. Therefore, there is necessarily a 
spanning tree $T_1$ of $G$ such that $e_1\in T_1$.

For contradiction, assume that there is a MCST $T_2$ such that
$e_1\notin T_2$. By the Exchange Lemma, there is an $e_2$ in $T_2$
such that $T_3=T_2\cup\{e_1\}-\{e_2\}$ is a spanning tree. But 
$c(e_1)<c(e_2)$, so $c(T_3)<c(T_2)$; that is, $T_2$ is not a minimum
cost spanning tree. We've found our contradiction; there cannot be a 
MCST that does not contain $e_1$. Clearly, there must me at least one
minimum cost spanning tree. We've shown that it contains $e_1$.
\end{proof}

\end{document}
