\documentclass{article}


\usepackage{fancyhdr}
\pagestyle{fancy}
\lhead{Problem 1.1 Sample}
\rhead{An introduction to the analysis of algorithms}
\renewcommand{\headrulewidth}{0.4pt}
\renewcommand{\footrulewidth}{0.4pt}

\newtheorem{inner}{Problem}
\newenvironment{prb}[1]
	{\renewcommand\theinner{#1}\inner}
	{\endinner}

\begin{document}

Let $O=A(I)$ be the output of $A$ on $I$, if it exists.
Let $\alpha_A$ be a pre-condition and $\beta_A$ a post-condition of A;
if $I$ satisfies the pre-condition we write $\alpha_A(I)$ and
if $O$ satisfies the post-condition we write $\beta_A(O)$.
Then, partial correctness of $A$ with respect to pre-condition
$\alpha_A$ and post-condition $\beta_A$ can be stated as:

\begin{equation}\label{eq1}
	(\forall I\in\mathcal{I}_A)
	[(\alpha_A(I)\wedge\exists O(O=A(I)))\rightarrow\beta_A(A(I))]
\end{equation}

\noindent which in words states the following: for any well formed 
input $I$, if $I$ satisfies the pre-condition and $A(I)$ terminates
or produces an output, i.e. $\exists O(O=A(I))$, then this output
satisfies the post-condition. Full correctness is (\ref{eq1})
together with the assertion that for all $I\in\mathcal{I}_A$, $A(I)$
terminates (and hence there exists an $O$ such that $O=A(I)$).

\begin{prb}{1.1}
Modify (\ref{eq1}) to express full correctness.
\end{prb}

\noindent {\bf Solution:}\break
We already have a full expression of partial correctness in $\ref{eq1}$.
We know that full correctness combines partial correctness with the
assertion that the algorithm is guaranteed to terminate. This assertion
is identical to:

\begin{equation}\label{eq2}
\forall I\in\mathcal{I}_A(\exists O(O=A(I)))
\end{equation}

Partial correctness guarantees that if $A$ is given a well-formed input
$I$ meeting the pre-condition $\alpha_A$ and terminates with result 
$O=A(I)$, then $O$ meets the post-condition $\beta_A$. Full correctness
then adds the guarantee of termination (i.e. the guarantee that $A$ will
terminate when given a well-formed input, producing result $O$). The result
is that $A$, when given any well-formed input $I\in\mathcal{I}_A$, will
terminate with result $O$, and moreover that if $I$ meets the pre-conditions
(i.e. if $\alpha_A(I)$) then $O$ will meet the post-conditions ($\beta_A(O)$).
This can all be written succinctly in a single equation, similar to 
(\ref{eq1}):

\begin{equation}\label{eq3}
	(\forall I\in\mathcal{I}_A)
	[\exists O(O=A(I))\wedge(\alpha_A(I)\rightarrow\beta_A(A(I)))]
\end{equation}

\end{document}
