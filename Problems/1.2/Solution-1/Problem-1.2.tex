\documentclass{article}


\usepackage{fancyhdr}
\usepackage{amssymb}
\pagestyle{fancy}
\lhead{An introduction to the analysis of algorithms\break Michael Soltys}
\rhead{Problem 1.2 Solution\break Ryan McIntyre}
\renewcommand{\headrulewidth}{0.4pt}
\renewcommand{\headheight}{24pt}
\pagenumbering{gobble}

\newtheorem{inner}{Problem}
\newenvironment{prb}[1]
	{\renewcommand\theinner{#1}\inner}
	{\endinner}
\newcommand{\Z}{\mathbb{Z}}
\newcommand{\N}{\mathbb{N}}

\begin{document}

$an^2+bn+c=\Theta(n^2)$, where $a>0$. To see this, note that 
$an^2+bn+c\leq(a+|b|+|c|)n^2$, for all $n\in\N$, and so 
$an^2+bn+c=O(n^2)$. On the other hand, 
$an^2+bn+c=a((n+c_1)^2-c_2)$ where $c_1=b/2a$ and 
$c_2=(b^2-4ac)/4a^2$, so we can find a $c_3$ and an $n_0$ so that
for all $n\geq n_0$, $c_3n^2\leq a((n+c_1)^2-c_2)$, and so
$an^2+bn+c\in\Omega(n^2)$.

\begin{prb}{1.2}
Find $c_3$ and $n_0$ in terms of $a$, $b$, and $c$. Then prove that
for $k\geq0$, $\sum_{i=0}^ka_in^i=\Theta(n^k)$; this shows the 
simplfying advantage of the Big O.
\end{prb}

\noindent{\bf Solution:}\break
Clearly, 
\begin{equation}\label{eq1}
an^2+bn+c\ge an^2-|b|n-|c|=n^2(a-|b|/n-|c|/n^2)
\end{equation} 
$|b|$ is finite, so $\exists n_b\in\N$ such that $|b|/n_b\le a/4$.
Similarly, $\exists n_c\in\N$ such that $|c|/n_c^2\le a/4$.
Let $n_0=max\{n_b,n_c\}$. For $n\ge n_0$, 
$a-|b|/n_0-|c|/n_0^2\ge a-a/4-a/4=a/2$. This, combined with (\ref{eq1}),
grants:
\begin{equation}\label{eq2}
\frac{a}{2}n^2\leq an^2+bn+c
\end{equation}
for all $n\ge n_0$. We need only to assign $c_3$ the value $a/2$ to complete
the proof that $an^2+bn+c\in\Omega(n^2)$.

Next we deal with the general polynomial with a positive leading coefficient.
Let $p(n)=\sum_{i=1}^ka_in^i=n^k\sum_{i=1}^k\frac{a_i}{n^{k-i}}$, 
where $a_k>0$. Clearly $p(n)\leq n^k\sum_{i=1}^k|a_i|$ for all $n\in\N$,
so $p(n)=O(n^k)$. Moreover, every $a_i$ is finite, so for each $i\in\N$
such that $0\le i\le k-1$, $\exists n_i$ such that 
$\frac{a_i}{n^{k-i}}\le a_k/2k$ for all $n\ge n_i$. Let $n_0$ be the maximum of
these $n_i$'s. $p(n)$ can be rewritten as 
$n^k(a_k+\sum_{i=0}^{k-1}\frac{a_i}{n^{k-i}})$, so 
$p(n)\ge n^k(a_k-\sum_{i=0}^{k-1}\frac{a_i}{n^{k-i}})$.
We have shown that for $n\ge n_0$, 
$\sum_{i=0}^{k-1}\frac{a_i}{n^{k-i}}\le a_k-k(a_k/2k)=a_k/2$, so let $c=a_k/2$.
For all $n\ge n_0$, $p(n)\ge(a_k-a_k/2)n^k=cn^k$. Thus, $p(n)=\Omega(n^k)$.

We have shown that $p(n)\in O(n^k)$ and $p(n)\in\Omega(n^k)$, so 
$p(n)=\Theta(n^k)$. 

\end{document}
