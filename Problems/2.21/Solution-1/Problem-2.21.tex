\documentclass{article}

\usepackage{fancyhdr}
\usepackage{amsmath,amssymb,amsthm}
\usepackage[utf8]{inputenc}
\usepackage[T1]{fontenc}
\usepackage{algorithm,algorithmic}
\pagestyle{fancy}
\lhead{An introduction to the analysis of algorithms\break Michael Soltys}
\rhead{Problem 2.21 Solution\break Ryan McIntyre}
\renewcommand{\headrulewidth}{0.4pt}
\renewcommand{\headheight}{24pt}
\newcommand{\opt}{\text{\sc Opt}}
\pagenumbering{gobble}

\newtheorem{innerthm}{Theorem}
\newenvironment{thm}[1]
	{\renewcommand\theinnerthm{#1}\innerthm}
	{\endinnerthm}

\newtheorem{innerlem}{Lemma}
\newenvironment{lem}[1]
	{\renewcommand\theinnerlem{#1}\innerlem}
	{\endinnerlem}

\newtheorem{innerprb}{Problem}
\newenvironment{prb}[1]
	{\renewcommand\theinnerprb{#1}\innerprb}
	{\endinnerprb}

\begin{document}

\begin{algorithm}
\caption{Job scheduling}%
\label{alg:jobs}%
\begin{algorithmic}[1]
\STATE Sort the jobs in non-increasing order of profits:
  $g_1\geq g_2\geq\ldots\geq g_n$ 
\STATE $d\longleftarrow\max_id_i$ 
\FOR{$t:1..d$} 
     \STATE $S(t)\longleftarrow 0$ 
\ENDFOR
\FOR{$i:1..n$} 
     \STATE Find the largest $t$ such that $S(t)=0$ and $t\leq d_i$,
            $S(t)\longleftarrow i$
\ENDFOR
\end{algorithmic}
\end{algorithm}

\begin{thm}{2.18}\label{thm}
The greedy solution to job scheduling is optimal\index{optimal!job
scheduling}. That is, the profit $P(S)$ of the schedule $S$ computed
by algorithm~\ref{alg:jobs} is as large as possible.
\end{thm}

\begin{lem}{2.19}\label{lem}
``$S$ is promising'' is an invariant for the (second) for-loop
in algorithm~\ref{alg:jobs}.   
\end{lem}

\begin{prb}{2.21}
Why does lemma~\ref{lem} imply theorem~\ref{thm}?
(Hint: this is a simple observation). 
\end{prb}

\begin{proof}
Assume $S_{i-1}$ (i.e. $S$ after $i-1$ iterations) is promising.

\noindent{\bf Case 1:} If task $i$ is not 
scheduled on iteration $i$, then there is no $S'$ that extends 
$S_{i-1}$ which schedules task $i$. But $S_{i-1}$ is promising, so 
there is an optimal $S'$ that extends $S_{i-1}$; clearly the $S'$ does
not schedule task $i$, so ``losing access'' to task $i$ does not 
change the fact that $S_i=S_{i-1}$ is promising.

\noindent{\bf Case 2:} If, on the other hand, task $i$ is scheduled at
time $t_i$, let $S_{i-1}'$ be the optimal extension of $S_{i-1}$. If
$S_{i-1}'$ schedules task $i$ at $t_i$, then we're done. If $t_i$ is
scheduled at a different time $t_0$, then $t_0<t_i$, as $t_i$ was the
latest available time for task $i$. So let $S_i'$ be $S_{i-1}'$, but
with the tasks at times $t_i$ and $t_0$ switched; we know task $i$ may
be moved to $t_i$ as $t_i$ is before $d_i$, and we know the task
scheduled at $t_i$ may be moved to $t_0$ because $t_0<t_i$.
We have found an extension of $S_i$ with the same cost as $S_{i-1}'$,
and $S_{i-1}'$ is optimal. Therefore, $S_i$ is promising.
\end{proof}

\end{document}
