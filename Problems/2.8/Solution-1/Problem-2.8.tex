\documentclass{article}

\usepackage{fancyhdr}
\usepackage{amsmath,amssymb,amsthm}
\usepackage[utf8]{inputenc}
\usepackage[T1]{fontenc}
\pagestyle{fancy}
\lhead{An introduction to the analysis of algorithms\break Michael Soltys}
\rhead{Problem 2.8 Solution\break Ryan McIntyre}
\renewcommand{\headrulewidth}{0.4pt}
\renewcommand{\headheight}{24pt}
\newcommand{\opt}{\text{\sc Opt}}
\pagenumbering{gobble}

\newtheorem{inner}{Problem}
\newenvironment{prb}[1]
	{\renewcommand\theinner{#1}\inner}
	{\endinner}
%\newenvironment{proof}
%{\noindent\textsc{Proof:}}{\hfill $\square$}

\begin{document}

\begin{prb}{2.8}
	Show that given a connected graph $G=(V,E)$, Kruskal's algorithm outputs a
	$T$ that is both connected and acyclic.
\end{prb}

\begin{proof}
We use the following loop invariant from the text:
\begin{equation}\label{eq1}
\textit{\emph{The edge set $T\cup\{e_{i+1},\ldots,e_m\}$ connects all
nodes in $V$.}}
\end{equation}

We start from the basis case: before the first iteration, $T_0$ is
the empty set ($i=0$). Since $G$ is connected, obviously 
$\{e_1,e_2,\ldots,e_m\}=E$ connects all nodes in $V$.

Next we prove induction. Assume that, after $i-1$ iterations,
$T_{i-1}\cup\{e_i,\ldots,e_m\}$ connects all nodes in $V$. On iteration
$i$, we have two cases:

\noindent{\bf Case 1:} $T_{i-1}\cup\{e_i\}$ has no cycle, so 
$T_i=T_{i-1}\cup\{e_i\}$. $T_i\cup\{e_{i+1},\ldots,e_m\}$ and 
$T_{i-1}\cup\{e_i,\ldots,e_m\}$ are the same set, $e_i$ has just
moved from the ``remaining'' edges to $T$. By the hypothesis, the 
latter edge set connects all nodes in $V$, so the prior must as well.

\noindent{\bf Case 2:} $T_{i-1}\cup\{e_i\}$ contains a cycle, so
$T_i=T_{i-1}$. Consider any two nodes $u,v\in V$. By the hypothesis,
there is a path from $u$ to $v$ consisting of edges in 
$T_{i-1}\cup\{e_i,\ldots,e_m\}$. If $e_i$ is not in this path,
then we're done; there is still a path between $u$ and $v$, as 
we've only lost access to $e_i$. If $e_i=(a,b)$ is in this path, we
can replace it with another path from $a$ to $b$; $e_i$ was in a cycle,
so another such path necessarily exists. 

We have found a path connecting arbitrary $u$ and $v$ in 
$T_i\cup\{e_i,\ldots,e_m\}$ given that one existed in 
$T_{i-1}\cup\{e_i,\ldots,e_m\}$, thereby completing the induction step
and proving that (\ref{eq1}) is a loop invariant.

Clearly, after all $i=m$ iterations, this loop invariant reads
``$T_i\cup\{e_{i+1},\ldots\}$ connects all nodes in $V$.'' But
$e_m$ was the last edge, so $\{e_{i+1},\ldots\}$ is the empty set.
Therefore, $T_m$ connects all nodes in $V$. By construction, 
$T_m$ cannot contain any cycles; any edge which would
have completed a cycle was simply not included. So, after $m$ iterations,
$T$ connects all nodes in $V$ and is acyclic - 
$T$ is a spanning tree of $G$.\qedhere 
\end{proof}

\end{document}
